\documentclass{article}
\usepackage[utf8]{inputenc}
\usepackage[spanish]{babel}
\usepackage{listings}
\usepackage{graphicx}
\graphicspath{ {images/} }
\usepackage{cite}

\begin{document}

\begin{titlepage}
    \begin{center}
        \vspace*{1cm}
            
        \Huge
        \textbf{Nociones de la memoria del computador}
            
        \vspace{0.5cm}
        \LARGE
        taller de memoria
            
        \vspace{1.5cm}
            
        \textbf{Daniela Rosa Villadiego Padilla}
         \newline  
         \vspace{3,5cm}
         
        \\Docente
         
        \newline
        \textbf{Augusto Enrique Salazar Jimenez}
        \vfill
            
        \vspace{0.8cm}
            
        \Large
        Despartamento de Ingeniería Electrónica y Telecomunicaciones\\
        Universidad de Antioquia\\
        Medellín\\
        Septiembre de 2020
            
    \end{center}
\end{titlepage}

\tableofcontents
\newpage
\section{¿Qué es la memoria del computador?} \label{contenido}
La memoria del computador es el dispositivo que retiene, memoriza y/o almacena los datos informáticos durante algún periodo de tiempo. Esta proporciona una de las principales funciones de la computación, que son: almacenamiento y conocimiento, a parte de ser uno de los principales componentes que tiene el computador ya que conecta a la unidad central de procesamiento con los dispositivos de entrada y salida de este. \cite{Wiki}
\section{Tipos de memoria que conozco.} \label{contenido}
•	Random Access Memory (RAM): La memoria RAM se encarga del almacenamiento temporal que guarda los programas y los datos que están siendo procesados, los datos solo permanecen almacenados en ella mientras la computadora se encuentre prendida, en el momento que el pc se apaga todos esos datos se pierden. La memoria RAM es fundamental para lograr una buena performance de nuestro equipo. 
\newline

•	Read Only Memory (ROM): La memoria ROM se encarga del almacenamiento de aplicaciones y/o datos permanentes o raramente alterados. La información generalmente es colocada en el chip de almacenamiento cuando es fabricado y el contenido de la ROM no puede ser alterado por un programa de usuario. Por ese motivo es una memoria sólo de lectura.
\newline

•	Disco Duro: Es el centro de almacenamiento de datos del computador. Aquí es donde se instala el software y donde se almacenan los documentos y todo tipo de archivos. El disco duro guarda y protege los datos a largo plazo, lo que significa que quedarán guardados incluso si se apaga el computador. \cite{GCFGLOBAL} \cite{Graciela}

\section{Descripción de la manera de cómo se gestiona la memoria en un computador.}\label{contenido}
Una vez que la memoria se encuentra en funcionamiento se carga en la misma un programa llamado BIOS (sistema básico de entrada y salida) el cual es un software que localiza y reconoce todos los dispositivos necesarios para cargar el sistema operativo en la memoria RAM, posee un componente de hardware y otro se software. El software brinda una interfaz generalmente de texto que permite configurar varias opciones del hardware instalado en el pc, de esta forma cada vez que el microprocesador quiera enviar o recibir una instrucción podrá encontrarlos simplemente buscando en su dirección o identificador cargado en la memoria RAM desde el momento en que es encendida la computadora. \cite{Augusto} \cite{Vikidia}

\section{¿Qué hace que una memoria sea más rápida que otra? ¿Por qué esto es importante?}\label{contenido}
La diferencia de velocidad se puede dar por el tipo de material con el que estén construidas, otro elemento vital de la memoria para conseguir el mejor rendimiento posible es la latencia. En términos generales, la latencia es el tiempo que transcurre desde que la memoria recibe un comando, hasta que lo ejecuta. Dicho en otras palabras, la latencia de la RAM por ejemplo es el intervalo de tiempo entre las dos acciones; cuanto más bajo sea este valor, mejor.
\newline 
\vspace{0,5cm}

Esto es importante ya que la velocidad de la memoria determinará la velocidad a la que la CPU pueda procesar los datos, cuanto mayor sea la velocidad más rápido podrá el sistema leer y escribir información en la memoria. \cite{tarjetas} \cite{grupo}
\newpage


\bibliographystyle{IEEEtran}
\bibliography{references.bib}

\end{document}